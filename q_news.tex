\documentclass{article}
\usepackage{graphicx} % Required for inserting images
\usepackage{hyperref}
\usepackage[section]{placeins}
\title{\underline {\textbf{Vacancy, Probability, and Information}}}
\author{Jaehah Shin}
\date{May 6th 2023}
\begin{document}
\maketitle
\begin{figure}[h]
    \centering
    \includegraphics[width = \linewidth]{info.jpeg}
    \caption{}
    \label{fig1}
    
\end{figure}

\section{About the Author}
My name is Jaehah Shin, and I am a first-year Engineering Science student. 
I was talking with one of the advisors from UTQC about quantum information at Starbucks when he told me how energy could affect the probability of outcome with the equation;
this reminds me of Boltzmann's distribution. During taking MSE 160, which is material science and engineering course, they teach how the imperfection in the crystal increases the metal's strength. 
One of the imperfections is related to Boltzmann's distribution.
I am fascinated about how I can connect what I learn from the school and quantum information.

\section{Vacancies}
Before talking about the vacancy, I will return to the point where I left in the" About the Author" section. 
Imperfection in crystal increases the metal's strength. 
There are four types in general when people describe imperfection: zero-dimensional imperfection, one-dimensional imperfection, two-dimensional imperfection, and three-dimensional imperfection. 
I will talk about the vacancy that is a zero-dimensional imperfection. 
In short, vacancy represents the missing atoms from the lattice site.
Vacancies are thermally generated, and this happens when thermal energy becomes greater than the binding energy. 
Does that mean at the point when thermal energy exceeds the binding energy, every atom in the lattice site is all coming out from where they are supposed to be? 
The answer is simple. No. Then how and why? 
Now, this is the time we should introduce the topic of energy and the probability of this. 
Figure 2 below shows a vacancy.
\begin{figure}[h]
    \centering
    \includegraphics[width = \linewidth]{vacancy.png}
    \caption{Vacancy}
    \label{fig:Vacancy}
\end{figure}

\section{Boltzmann distribution}
To explain why vacancies don't take over the lattice when the thermal energy exceeds the binding energy, we should consider Boltzmann's distribution. 
The following equation explains the relationship between temperature and the probability of vacancies being generated. 
\LARGE
$\frac{N_v}{N} = e^\frac{-Q_v}{KT}\ $     
\\
\normalsize
\begin{itemize}
    \item $ N_v$ is number of vacancies in the crystal 
    \item $ N$ is number of cites of where atoms occupies in the lattice cite. 
    \item $-Q_v$ is energy required to form vacancy which is binding energy.
    \item $K$ is Boltzmann's constant 
    \item $T$ is temperature in Kelvin. 
\end{itemize}

This equation states that when temperature increases, the number of vacancies increase exponentially. 
Figure 3 shows the probability of number of atoms in energy level depending on the temperature.
\\
\begin{figure}
    \centering
    \includegraphics[width = \linewidth]{boltzmannnnn.JPG}
    \caption{Possibility of site of atoms depending on the temperature}
    \label{fig:l}
\end{figure}
% idk this supposed to put into the specific area, but i deosnt seem like moving...
This demonstrates when temperature increases, as all atoms are populated on every energy level evenly. Therefore, the possibility of the vacancy being generated increases with temperature increases. 
Ideally, under infinite temperature, as the atoms are all distributed in every energy level, there must be $N_v$ = $N$. 
This is the only way when every atom in the lattice site comes out from where they are supposed to be.
This is why this never happens in real life, as the infinite temperature is unreachable. 
This answers the question I discussed in the section on "Vacancy." 
We saw how the thermal energy which is generated by temperature affects the probability of making a vacancy. 
Generally, energy is the one that can change the probability, which will be discussed in the following section.
Finally, let's discuss how quantum mechanics plays a role in this world. 
\section{Information in Quantum Mechanics}
What is information in the quantum world? 
Quantum information is a set of distinguishable outcomes.
There are sets of operators that can change the quantum information by changing the probability of the outcome.
Boltzmann's distribution demonstrates that when thermal energy is applied to the molecule, 
then its speed increases, which results in the probability distribution particle's velocity smoothens out through all the energy values.  
% write about boltzmann distribution with more clear definition and more clear description/ 

\section{The relationship between vacancy and quantum information}
Those vacancies can be served as qubits, the basic unit of information in a quantum computer. 
"Semiconductors are the material behind the brains in cell phones, computers, medical equipment, and more."[1] 
Imperfections generally enhance the strength of metals. 
Vacancies, as imperfections themselves, might be expected to contribute to this strength improvement. 
However, contrary to expectations, vacancies do not increase the strength of metals.
Also, generally, in terms of the applications for the semiconductor, the existence of those defects in the crystal is the one that scientists try to avoid since this can be the reason to lower the performance of the semiconductor. 
However, in recent studies, it turns out that "certain types of vacancies in silicon carbide and other semiconductors show promise for the realization of qubits in quantum devices." 

Now, the current research explains that they need to learn more to customize vacancies for their specific and desired purposes. 

\section{Application of Vacancy for Quantum Information}
% section 6
This section summarizes paper[4] on how the vacancy applies in the real world for quantum information. 
Single Silicon Vacancy Centers are applied in 10nm Diamonds for Quantum Information. \\
This diamond is artificially made under high-pressure and temperature synthesis and contains the optically active, single silicon-vacancy color center.
Then, the special method will prepare the sample with that diamond on the surface. 
The research lab confirmed the existence of optically active color centers within individual nanocrystals by comparing atomic-force microscope images with confocal optical images. 
Additionally, through second-order correlation measurements, the lab demonstrated that these nanodiamonds exhibit single-photon emission statistics. 
These color centers exhibit steady emission without blinking, with a narrow spectral width and precise positioning of the zero-phonon line, providing evidence for the excellent quality of the produced nanodiamonds.
\\
As "the nanodiamonds demonstrate high single-photon brightness (10\textsuperscript{5} cps) at room temperature"[4].
This also indicates that this is suitable for quantum information as this has high brightness. 
Also, this has "narrowband luminescence and extremely stable ZPL position"[4]. 
Therefore, this has a low number of defects which differ from SiV centers s in the nanodiamonds grown.\\

Also, color centers in diamonds can generate single photons at room temperature and access electron spin; therefore, this brings lots of attention to the research area.  
\section{Conclusion}
In conclusion, quantum information in the quantum world refers to a set of distinguishable outcomes that can be altered using operators.
Vacancies, typically considered imperfections, show promise as qubits for quantum information processing. Specifically, certain types of vacancies in semiconductors like silicon carbide have potential for qubit realization.
Nanodiamonds containing single silicon vacancy centers exhibit high brightness, stable emission, and access to electron spin, making them suitable for quantum information processing. 
Further research is needed to customize vacancies and explore their applications in quantum computing.
More specifically, research will be more focused on making the computation much faster and simulating more defects 
and determining the best defects for different applications. \\
Before finishing off, I would like to share a great video that I found. \\
\href{https://www.youtube.com/watch?v=KZIyG9II514}{Click this to see the video!} \\
This is the video made by the University of Chicago, and they have tracked the pairing of individual vacancies 
into a divacancy (paired vacancy) under a very high temperature. However, this yields a low probability of the outcome.
Therefore, they found out that more silicon vacancies relative to carbon vacancies are needed to form more divacancies under the 
heat treatment.


\section{Citation}
Figure 1: “Quantum Information,” Wikipedia, 07-Apr-2023. [Online]. Available: https://en.wikipedia.org/wiki/Quantum information. [Accessed: 06-May-2023]. 
Figure 2: “Material defects,” the first year engineer, 03-Sep-2016. [Online]. Available: https://firstyearengineer.com/material-science/introduction/defects/. [Accessed: 06-May-2023]. 
[1]“How to transform vacancies into Quantum Information,” ScienceDaily, 15-Dec-2021. [Online]. Available: https://www.sciencedaily.com/releases/2021/12/211215204109.htm. [Accessed: 17-Mar-2023].  
[2] E. M. Y. Lee, A. Yu, J. J. de Pablo, and G. Galli, "Stability and molecular pathways to the formation of spin defects in Silicon Carbide," Nature News, 03-Nov-2021. [Online]. Available: \url{https://www.nature.com/articles/s41467-021-26419-0}. [Accessed: 26-Mar-2023]. 
[3]Single Silicon Vacancy Centers in 10 nm Diamonds for Quantum Information Applications
Stepan V. Bolshedvorskii, Anton I. Zeleneev, Vadim V. Vorobyov, Vladimir V. Soshenko, Olga R. Rubinas, Leonid A. Zhulikov, Pavel A. Pivovarov, Vadim N. Sorokin, Andrey N. Smolyaninov, Liudmila F. Kulikova, Anastasiia S. Garanina, Sergey G. Lyapin, Viatcheslav N. Agafonov, Rustem E. Uzbekov, Valery A. Davydov, and Alexey V. Akimov
ACS Applied Nano Materials 2019 2 (8), 4765-4772
DOI: 10.1021/acsanm.9b00580

\end{document}
