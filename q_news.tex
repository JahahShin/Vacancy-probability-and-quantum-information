\documentclass{article}
\usepackage{graphicx} % Required for inserting images
\usepackage{hyperref}
\usepackage[section]{placeins}
\title{\underline {\textbf{Vacancy, Probability, and Information}}}
\author{Jaehah Shin}
\date{May 6th 2023}
\begin{document}
\maketitle
\begin{figure}[h]
    \centering
    \includegraphics[width = \linewidth]{info.jpeg}
    \caption{Brief Image for Quantum Information}
    \label{fig:z}
\end{figure}

\section{About the Author}
I am the person who is first year in Engineering Science. I was talking with one of the advisors from UTQC about quantum information at the Starbucks, when he told me about how energy can affect to the probability of outcome with the equation, this reminds me of Boltzmann's distribution. 
During taking MSE 160 which is material science and engineering course, they teach how the imperfection in the crystal increases the metal strength. One of the imperfection is related to the Boltzmann's distribution. I am fascinated about the fact that how I can connect what I learn from the school and quantum information. I can't wait to inform other people what I found out, and how I connect those seemingly two entirely different concepts. 
\section{Vacancies}
Before talking about vacancy, I will go back to the point where I left in "About the Author" section. Imperfection in crystal increased the metal strength. There are four types in general when people describe the imperfection which are zero - dimensional imperfection, one -dimensional imperfection, two - dimensional imperfection, and three dimensional imperfection. 

What I am going to talk about is the vacancy which is a zero - dimensional imperfection. In short, vacancy represent the missing atoms from the lattice cite. Vacancies are thermally generated, and this happens when thermal energy became greater than the binding energy. Does that mean at the point when thermal energy exceeds the binding energy, every atoms in lattice cite are all coming out from where are they supposed to be? The answer is simple. No. Then how and why? Now, this is the time we should introduce the topic energy and the probability in to this.
Below figure shows the vacancy.
\begin{figure}[h]
    \centering
    \includegraphics[width = \linewidth]{vacancy.png}
    \caption{Vacancy}
    \label{fig:Vacancy}
\end{figure}
\section{Boltzmann distribution}
This might be surprised you that the section is not named as "energy and the probability." However, Boltzmann distribution is related to the energy and possibilities in the way that you may have never thought of. 

Back to the point where I left in the section 2, this is the time I can finally reveal the fact why all of the atoms are not coming out when the thermal energy exceeds the binding energy. 
The following equation explains pretty much everything about the relationship between temperature and the probability of vacancies being generated. 
\LARGE
$\frac{N_v}{N} = e^\frac{-Q_v}{KT}\ $     
\\
\normalsize
\begin{itemize}
    \item $ N_v$ is number of vacancies in the crystal 
    \item $ N$ is number of cites of where atoms occupies in the lattice cite. 
    \item $-Q_v$ is energy required to form vacancy which is binding energy.
    \item $K$ is Boltzmann's constant 
    \item $T$ is temperature in Kelvin. 
\end{itemize}

As you can easily tell from this equations, this equation tells that when temperature increases, the number of vacancies increase exponentially. \\ The diagram shows in next page is the probability of number of atoms in energy level depending on the temperature.
% the following line just doesnt inser the image properly, and dk how to do it.
\\

\begin{figure}
    \centering
    \includegraphics[width = \linewidth]{boltzmannnnn.JPG}
    \caption{Possibility of Site of Atoms depending on Temperature}
    \label{fig:l}
\end{figure}
% idk this supposed to put into the specific area, but i deosnt seem like moving...
This demonstrates when temperature increases, as all atoms are populated on every energy level evenly, therefore, the possibility of vacancy being generated is increasing as increasing of temperature. Ideally, under the condition of infinite temperature, as the atoms are all distributed in every energy level, there must be $N_v$ = $N$. This is the only way that when every atoms in lattice cite are all coming out from where are they supposed to be. And, this is the reason why this never happens in the real life, as infinite temperature is not reachable in the real life. Now, I hope this answers the question that I discussed in the section of "Vacancy". We saw how the thermal energy which is generated by temperature affects the probability in making vacancy. 
Generally, energy is the one that can change the probability in the world, which will be discussed in the following section.
Finally, this is time to discuss about how the quantum mechanics play a role in this world. 
\section{Information in Quantum Mechanics}
What is an information in quantum world? 
First, quantum information is a set of distinguishable outcomes.
Second, quantum information is a set of operators that connect those
outcomes.
There is only one way to change the probability of outcome in quantum mechanics, which is when the energy interrupt to the probability. 
Boltzmann's distribution demonstrates that when the thermal energy is applied to the molecular, then there speed of the particle is getting increased, which results in the more possibility that particle is more spreads through all of the energy levels. 

\section{The relationship between vacancy and quantum information}
In section 2, we discussed about the vacancy. Vacancy can be generated by thermal energy, and scientist can increase exponentially the possibilities by increasing the temperature. And, now it turns out that those vacancies can be served as qubits which are the basic unit of quantum technology in the world. "Semiconductors are the material behind the brains in cell phones, computers, medical equipment and more."[1] In general, imperfections mostly help metals to increase the strength. As vacancies are imperfections, it seems like it also should help to increase the strength of the metal, but in fact, it doesn't increase the strength of the metal.
Also, generally, in terms of the applications for the semiconductor, the existence of those defects in the crystal is the one that scientists try to avoid, since this can be the reason to lower the performance of the semiconductor. 
However, in the recent studies, it turns out that "certain types of vacancies in silicon carbide and other semiconductors shows promise for the realization of qubits in quantum devices." 

Now, the important aspect in this they need to learn more so that they can customize vacancies for their specific and desired purposes. 

\section{Real Application of Vacancy on the Quantum Information}
This section summarizes the paper[4] so that people can see how the vacancy applies in the real world for the quantum information. 
There is a application of single Silicon Vacancy Centers in 10nm Diamonds for Quantum Information. \\
This diamond is artificially made under high-pressure and temperature synthesis and contains the optically active, single silicon-vacancy colour center. Then prepare the sample that has that diamond on the surface by using the special method. \\

"By correlating atomic-force microscope images and confocal optical images, we verified the presence of optically active color centers in single nanocrystals, and using second-order correlation measurements, we proved the single-photon emission statistics of these nanodiamonds. These color centers have non blinking, spectrally narrow emission with narrow distribution of spectral width and positions of zero-phonon line, thus proving the high quality of the nanodiamonds produced."[4]\\

\\
As "the nanodiamonds demonstrate high single-photon brightness (>100 000 cps) at room temperature."[4] This also indicates that this is suitable for quantum information as this has high brightness. Also. this has "narrowband luminescence and extremely stable ZPL position"[4], therefore, this has low number of deffects which differ from SiV centers s in the nanodiamonds grown.\\

Also, colour centers in diamond has the possibility of generating single photons at room temperature, and access to electron spin; therefore, this brings lots of attention for the research area. 
\section{Conclusion}
In conclusion, vacancy has a significant role in terms of quantum information. To change the probability, the energy is needed. The application of the vacancy towards the quantum information is still being researched, therefore, more information will be available as more research is done. 
Author is also looking for more about the application of the vacancy towards the quantum information. 

Before finishing off, I would like to share a great video that I found. 
\href{https://www.youtube.com/watch?v=KZIyG9II514}{Click This!} 

\section{Citation}
Figure 1: “Quantum Information,” Wikipedia, 07-Apr-2023. [Online]. Available: https://en.wikipedia.org/wiki/Quantum information. [Accessed: 06-May-2023]. 

Figure 2: “Material defects,” the first year engineer, 03-Sep-2016. [Online]. Available: https://firstyearengineer.com/material-science/introduction/defects/. [Accessed: 06-May-2023]. 

[1]“How to transform vacancies into Quantum Information,” ScienceDaily, 15-Dec-2021. [Online]. Available: https://www.sciencedaily.com/releases/2021/12/211215204109.htm. [Accessed: 17-Mar-2023].  \\


[2]E. M. Y. Lee, A. Yu, J. J. de Pablo, and G. Galli, “Stability and molecular pathways to the formation of spin defects in Silicon Carbide,” Nature News, 03-Nov-2021. [Online]. Available: https://www.nature.com/articles/s41467-021-26419-0. [Accessed: 26-Mar-2023]. 
[3]Single Silicon Vacancy Centers in 10 nm Diamonds for Quantum Information Applications
Stepan V. Bolshedvorskii, Anton I. Zeleneev, Vadim V. Vorobyov, Vladimir V. Soshenko, Olga R. Rubinas, Leonid A. Zhulikov, Pavel A. Pivovarov, Vadim N. Sorokin, Andrey N. Smolyaninov, Liudmila F. Kulikova, Anastasiia S. Garanina, Sergey G. Lyapin, Viatcheslav N. Agafonov, Rustem E. Uzbekov, Valery A. Davydov, and Alexey V. Akimov
ACS Applied Nano Materials 2019 2 (8), 4765-4772
DOI: 10.1021/acsanm.9b00580

\end{document}
